\documentclass[10pt,conference]{IEEEtran}

\ifCLASSINFOpdf
	\usepackage[pdftex]{graphicx}
	%\graphicspath{{./figs/}}
	\DeclareGraphicsExtensions{.pdf,.jpeg,.png}
\else
	\usepackage[dvips]{graphicx}
	%\graphicspath{{./figs/}}
	\DeclareGraphicsExtensions{.eps}
\fi

\usepackage[cmex10]{amsmath}
\usepackage[tight,footnotesize]{subfigure}
\usepackage{xcolor}
\usepackage[lined,ruled]{algorithm2e}

\usepackage[latin1]{inputenc}
\usepackage{tikz}
\usetikzlibrary{shapes}
\usetikzlibrary{arrows}

\usepackage[]{algorithm2e}

\newtheorem{property}{Property}
\newtheorem{proposition}{Proposition}
\newtheorem{theorem}{Theorem}
\newtheorem{conjecture}{Conjecture}
\newtheorem{question}{Question}


\newtheorem{definition}{Definition}
\newtheorem{corollary}{Corollary}

\makeatletter
\pgfdeclareshape{datastore}{
\inheritsavedanchors[from=rectangle]
\inheritanchorborder[from=rectangle]
\inheritanchor[from=rectangle]{center}
\inheritanchor[from=rectangle]{base}
\inheritanchor[from=rectangle]{north}
\inheritanchor[from=rectangle]{north east}
\inheritanchor[from=rectangle]{east}
\inheritanchor[from=rectangle]{south east}
\inheritanchor[from=rectangle]{south}
\inheritanchor[from=rectangle]{south west}
\inheritanchor[from=rectangle]{west}
\inheritanchor[from=rectangle]{north west}
\backgroundpath{
    %  store lower right in xa/ya and upper right in xb/yb
\southwest \pgf@xa=\pgf@x \pgf@ya=\pgf@y
\northeast \pgf@xb=\pgf@x \pgf@yb=\pgf@y
\pgfpathmoveto{\pgfpoint{\pgf@xa}{\pgf@ya}}
\pgfpathlineto{\pgfpoint{\pgf@xb}{\pgf@ya}}
\pgfpathmoveto{\pgfpoint{\pgf@xa}{\pgf@yb}}
\pgfpathlineto{\pgfpoint{\pgf@xb}{\pgf@yb}}
 }
}
\makeatother

\newcommand{\riham}[1]{{\color{red}{#1}}}
\newcommand{\james}[1]{{\color{blue}{#1}}}


\begin{document}

\title{CS512 FUN Projects - Fall 2015}
\author{
\IEEEauthorblockN{James Abello}
\IEEEauthorblockA{DIMACS Center, Rutgers University\\ Piscataway, NJ, USA\\
Email: abello@dimacs.rutgers.edu}
%\and
%\IEEEauthorblockN{Alan Turing}
%\IEEEauthorblockA{Rutgers University\\
% Piscataway, NJ, USA\\
% Email: alan1936@cs.rutgers.edu}
%\and
%%\IEEEauthorblockN{Third group member}
%\IEEEauthorblockA{Rutgers University\\
%\Piscataway, NJ, USA\\
%Email: Third_group_member@scarletmail.rutgers.edu}
}

\maketitle
\begin{abstract}
\textnormal{
Insert your Project Abstract here. It must have a strong algorithmic component and it must be FUN, i.e. Feasible, Useful, and Novel. You can refer to the Abstract of a sample research paper on algorithms to follow as a model.
}
\end{abstract}
%\onecolumn \maketitle \normalsize \vfill

\IEEEpeerreviewmaketitle
%%%%%%%%%%%%%%%%%%%%%%%%%%%%%%%%%%%%%%%%%%%%%%%%%%%%%%%%%%%%%%%%%%%%%%%%%%%%%%%%%%%%%%%%%%%%%%%%%%%%%%%%%
\section{Project Description}\label{sec:1. Project Description}
%%%%%%%%%%%%%%%%%%%%%%%%%%%%%%%%%%%%%%%%%%%%%%%%%%%%%%%%%%%%%%%%%%%%%%%%%%%%%%%%%%%%%%%%%%%%%%%%%%%%%%%%%
\textnormal{
Insert your overall project description here. Specify the project type according to the posted sakai announcement.What is it that you are proposing?. Why is it useful?. Is it feasible to complete your project within the semester?.  Is it a novel idea?  What are the main stumbling blocks? What is the timeline for your project progress?. How are you planning to reach the major milestones?. 
}

The project has four stages: Gathering, Design, Infrastructure Implementation, and User Interface.

%\subsection{Stage1 - The Requirement Gathering Stage. } \label{sec:1	Requirement Gathering Stage. } 
%\textnormal{
Get a realistic project idea that includes potential real world scenarios,
with a description of the different user types along with their interactions with the system 
as well as the system feedback to them, according to their information needs. 
This stage also requires the specification of the different constraints and restrictions
that need to be enforced depending on the different types of user (system interactions). 
The deliverables for this stage include the following items:
} 

\begin{itemize} 
\item{A general description (in plain English) of your project's deliverables (understandable by computer illiterate users).} 

\item{ A specific description of at least three types of users. }
	
\item{ A description of detailed real world scenarios (at least 2 scenarios) representing those typical interactions between the different user types and the system (including inputs and outputs and data types).}
	

\item{A description of detailed real world scenarios (at least 2 scenarios) representing those typical interactions between the different user types and the system (including inputs and outputs and data types). }

\item{A detailed time line for completion of the major implementaion stages together with the division of labor including testing, documentation, evaluaton, project report, and power point presentation.}

\end{itemize}

Please insert your deliverables for Stage1 as follows:

\begin{itemize} 
\item{The general system description: } 
Please insert the system description here.
\item{The three types of users (grouped by their data access/update rights): }
Please insert the users types in here, as follows:
\item{The user's interaction modes: }
Please insert the user's interaction modes here.
\item{The real world scenarios: }
Please insert the real world scenarios in here, as follows.
	\begin{itemize} 
	\item{Scenario1 description: }
	Please insert Scenario1 description here.
	\item{System Data Input for Scenario1: }
	Please insert System Data Input for Scenario1 here.
	\item{Input Data Types for Scenario1: }
	Please insert Input Data Types for Scenario1 here.
	\item{System Data Output for Scenario1: }
	Please insert System Data Output for Scenario1 here.
	\item{Output Data Types for Scenario1: }
	Please insert Output Data Types for Scenario1 in here.
	\item {Please repeat that pattern for each scenario (at least 2 scenarios per user).}
	\end{itemize}
Please repeat that pattern for each user type.
\item{Project Time line and Divison of Labor.}
Please insert here the time line and the corresponding implementation tasks.
\end{itemize}


\subsection{Stage1 - The Requirement Gathering Stage. }\label{sec:1 Requirement Gathering Stage. }
%%%%%%%
\textnormal{
Get a realistic project idea that includes potential real world scenarios,
with a description of the different user types along with their interactions with the system 
as well as the system feedback to them, according to their information needs. 
This stage also requires the specification of the different constraints and restrictions
that need to be enforced depending on the different types of user (system interactions). 
The deliverables for this stage include the following items:
} 

\begin{itemize} 
\item{A general description (in plain English) of your project's deliverables (understandable by computer illiterate users).} 

\item{ A specific description of at least three types of users. }
	
\item{ A description of detailed real world scenarios (at least 2 scenarios) representing those typical interactions between the different user types and the system (including inputs and outputs and data types).}
	

\item{A description of detailed real world scenarios (at least 2 scenarios) representing those typical interactions between the different user types and the system (including inputs and outputs and data types). }

\item{A detailed time line for completion of the major implementaion stages together with the division of labor including testing, documentation, evaluaton, project report, and power point presentation.}

\end{itemize}

Please insert your deliverables for Stage1 as follows:

\begin{itemize} 
\item{The general system description: } 
Please insert the system description here.
\item{The three types of users (grouped by their data access/update rights): }
Please insert the users types in here, as follows:
\item{The user's interaction modes: }
Please insert the user's interaction modes here.
\item{The real world scenarios: }
Please insert the real world scenarios in here, as follows.
	\begin{itemize} 
	\item{Scenario1 description: }
	Please insert Scenario1 description here.
	\item{System Data Input for Scenario1: }
	Please insert System Data Input for Scenario1 here.
	\item{Input Data Types for Scenario1: }
	Please insert Input Data Types for Scenario1 here.
	\item{System Data Output for Scenario1: }
	Please insert System Data Output for Scenario1 here.
	\item{Output Data Types for Scenario1: }
	Please insert Output Data Types for Scenario1 in here.
	\item {Please repeat that pattern for each scenario (at least 2 scenarios per user).}
	\end{itemize}
Please repeat that pattern for each user type.
\item{Project Time line and Divison of Labor.}
Please insert here the time line and the corresponding implementation tasks.
\end{itemize}



\subsection{Stage2 - The Design Stage. }\label{sec: 2:The Design Stage.}
%%%%%%%%%%%%%%%%%%%%%%%%%%%%%%%%%%%%%%%%%%%%%%%%%%%%%%%%%%%%%%%%%%%%%%%%%%%%%%%%%%%%%%%%%%%%%%%%%%%%%%%%%%
\textnormal{
Transform the project requirements into a system flow diagram, specifyng the different algorithms, data types and structures required for processing and their associated operations.  
The deliverables for this stage include the system flow diagram containing a graphical representation and  textual descriptions of the corresponding data trasnformations, high level pseudo code of the overall system operation, and overall system time and space complexity.
}

%\begin{itemize} 
%\item{ }
%A brief textual description of the overall flow diagram (along with its functional operation in the different user scenarios described in the first stage of the project).
%\item{ }
%A specification of each algorithm and associated data structures together with its entities, attributes, and operations ( include an English description of how they relate to your user scenario(s)).

%\end{itemize}
Please insert your deliverables for Stage2 as follows:
\begin{itemize} 
\item{  Short Textual Project Description. }
Please insert here the flow diagram textual description here together with its overall time and space complexity.
\item{ Flow Diagram. }
Please insert your system Flow Diagram here.
\item{ High Level Pseudo Code System Description. }
Please insert high level pseudo-code describing the major system modules as per your flow diagram.
\item{Algorithms and  Data Structures. }
Please insert a brief description of each major Algorithm and its associated data structures here.
\end{itemize}

\begin{itemize} 
\item{  Flow Diagram Major Constraints.}
Please insert here the integrity constraints:
	\begin{itemize} 
	\item{ Integrity Constraint. }
	Please insert the first integrity constraint in here together with its description and justification. 
	\end{itemize}
Please repeat the pattern for each integrity constraint.
\end{itemize}



\subsection{Stage3 - The Implementation Stage. }\label{sec: 3 The Implementation Stage.}
%%%%%%%%%%%%%%%%%%%%%%%%%%%%%%%%%%%%%%%%%%%%%%%%%%%%%%%%%%%%%%%%%%%%%%%%%%%%%%%%%%%%%%%%%
\textnormal{
Specify the language and programming environemnt you used for your implementation.
%Building the corresponding relational tables, according to the proposed ER model described in the previous phase %enforcing the different integrity constraints.  
The deliverables for this stage include the following items:
\begin{itemize} 
\item{}
Sample small data snippet. 
%The SQL tables that represent the ER project model, along with at least 3-5 rows of concrete data per table.
\item{}
Sample small output
%The normalization steps for each table, along with explanations/justifications of each normalization step.
\item{}
Working code
%The SQL table after the normalization steps (showing all table attributes).
\item{}
Demo and sample findings
%The SQL statements used to create the SQL tables, including the required triggers as well as the integrity constraints. At %least 2 triggers and 2 of each of the following constraint types have to exist in the project tables overall: 
\begin{itemize} 
\item{}
	Data size: In terms of  RAM size;  Disk Resident?; Streaming ?;  
\item{}
	List the most interestng findings in the data if it is a Data Exploration Project. For other project types consult with your project supervisor what the corresponding outcomes shall be. Concentrate on demonstrating the Usefuness and Novelty of your application.
%Whether some users will be denied access and/or updates to some data according to their roles (for example: student1 %can not access other students' ' grades, so a violation error pops up upon that action. Another example: a sales person %can see an item price, but can not change it, since only a manger can, also a violation error pops up upon that update %attempt).
\end{itemize}
\end{itemize}
}


\subsection{Stage4 -User Interface. }\label{sec: 4. User Interface.}
%%%%%%%%%%%%%%%%%%%%%%%%%%%%%%%%%%%%%%%%%%%%%%%%%%%%%%%%%%%%%%%%%%%%%%%%%%%%%%%%%%%%%%%%%%%%%%%%%%%%%%%%%%
\textnormal{
Describe a User Interface (UI) to your application along with the related information that will be shown on each interface view (How users will query or navigate the data and view the query or navigation results). The emphasis should be placed on the process a user needs to follow in order to meet a particular information need in a user-friendly manner.
The deliverables for this stage include the following items :
}
\begin{itemize} 
\item{The modes of user interaction with the data (text queries, mouse hovering, and/or mouse clicks ?).} 
\item{The error messages that will pop-up when users access and/or updates are denied   }
\item{The information messages or results that wil pop-up in response to user interface events. }
	
\item{ The error messages in response to data range constraints violations.}
	
\item{ The interface mechanisms that activate different views in order to facilitate data accesses, according to users'  needs. }
	
\item{Each view created must be justified. Any triggers built upon those views should be explained and justified as well. At least one project view should be created with a justification for its use. }	
\end{itemize}

Please insert your deliverables for Stage4 as follows:
\begin{itemize} 
\item{The initial statement to activate your application with the corresponding initial UI screenshot}
	
\item{Two different  sample navigation user paths through the data exemplifying the different modes of interaction and the corresponding screenshots. }
\item{}
	The error messages popping-up when users access and/or updates are denied (along with explanations and examples):
	\begin{itemize} 
	\item{The error message: }
	\item{The error message explanation (upon which violation it takes place): }
	Please insert the error message explanation in here.
	\item{The error message example according to user(s) scenario(s): }
	Please insert the error message example in here.
	 \end{itemize}
\item{}
	The information messages or results that pop-up in response to user interface events.
	\begin{itemize} 
	\item{The information message: }
	Please insert the error message in here.
	\item{The information message explanation and the corresponding event trigger }
	\item{The error message example in response to data range constraints and the coresponding user's scenario }
	Please insert the error message example in here.
	 \end{itemize}
\item{}
	The  interface mechanisms that activate different views.
	\begin{itemize} 
	\item{The interface mechanism: }
	Please insert the interface mechanism here.
	 \end{itemize}
\end{itemize}



\section{Project Highlights.}\label{sec:7. Project Highlights.}
%%%%%%%%%%%%%%%%%%%%%%%%%%%%%%%%%%%%%%%%%%%%%%%%%%%%%%%%%%%%%%%%%%%%%%%%%%%%%%%%%%%%%%%%%%%%%%%%%%%%%%%%%%
\textnormal{
\begin{itemize} 
\item{}
Only working applications will be acceptable at project completion. A running demo shoul be presented to your project advisor at a date to be specified after the second midterm. A version of your application shall be installed in a machine to be specifed later during the semester. Your final submissiom package will also include a final LaTeX report modeled after this document, as well as a Power Point Presentation.
\item{}
The presentation (7 to 8 minutes) should include at least the following items (The order of the slides is important):
\begin{enumerate}
\item{}
Title: Project Names (authors and affiliations)
\item{}
Project Goal
\item{}
Outline of the presentation
\item{}
Description
\item{}
Pictures are essential. Please include Interface snapshots exemplyfing tthe different modes of users's interaction.
\item{}
Project Stumbling Blocks
\item{}
Data collection, Flow Diagram, Integrity Constraints
\item{}
Sample Findings
\item{}
Future Extensions
\item{}
Acknowledgements
\item{}
References and Resources used(libraries, languages, web resources)
\item{}
Demo(3 minutes)
\end{enumerate}
Please follow the sample presentation mock up that is posted on Sakai.
\item{}
By Dec 1 your group should have completed the final submission. This includes a presentation (7 to 8 minutes) to your project advisor as well as a convincing  demo of your project functionalities (3 minutes): every group member should attend the demo (and presentation) indicating clearly  and specifically his/her contribution to the project.  This wil allow us to evaluate all students in a consistent and fair manner.
\item{}
Thank you, and best of luck!
\end{itemize}
}


\bibliographystyle{IEEEtran}
%\bibliography{IEEEabrv,bib_queyroi_abello2013}
%\bibliography{bib_queyroi_abello2013}

\end{document}


