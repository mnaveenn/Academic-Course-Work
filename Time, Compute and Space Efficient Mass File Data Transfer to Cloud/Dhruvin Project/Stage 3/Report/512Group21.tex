\documentclass[10pt,conference]{IEEEtran}

\ifCLASSINFOpdf
	\usepackage[pdftex]{graphicx}
	%\graphicspath{{./figs/}}
	\DeclareGraphicsExtensions{.pdf,.jpeg,.png}
\else
	\usepackage[dvips]{graphicx}
	%\graphicspath{{./figs/}}
	\DeclareGraphicsExtensions{.eps}
\fi

\usepackage[cmex10]{amsmath}
\usepackage[tight,footnotesize]{subfigure}
\usepackage{xcolor}
\usepackage[lined,ruled]{algorithm2e}

\usepackage[latin1]{inputenc}
\usepackage{tikz}
\usetikzlibrary{shapes}
\usetikzlibrary{arrows}
\usepackage{listings}

\usepackage[]{algorithm2e}

\newtheorem{property}{Property}
\newtheorem{proposition}{Proposition}
\newtheorem{theorem}{Theorem}
\newtheorem{conjecture}{Conjecture}
\newtheorem{question}{Question}
\newtheorem{definition}{Definition}
\newtheorem{corollary}{Corollary}

\makeatletter
\pgfdeclareshape{datastore}{
\inheritsavedanchors[from=rectangle]
\inheritanchorborder[from=rectangle]
\inheritanchor[from=rectangle]{center}
\inheritanchor[from=rectangle]{base}
\inheritanchor[from=rectangle]{north}
\inheritanchor[from=rectangle]{north east}
\inheritanchor[from=rectangle]{east}
\inheritanchor[from=rectangle]{south east}
\inheritanchor[from=rectangle]{south}
\inheritanchor[from=rectangle]{south west}
\inheritanchor[from=rectangle]{west}
\inheritanchor[from=rectangle]{north west}
\backgroundpath{
    %  store lower right in xa/ya and upper right in xb/yb
\southwest \pgf@xa=\pgf@x \pgf@ya=\pgf@y
\northeast \pgf@xb=\pgf@x \pgf@yb=\pgf@y
\pgfpathmoveto{\pgfpoint{\pgf@xa}{\pgf@ya}}
\pgfpathlineto{\pgfpoint{\pgf@xb}{\pgf@ya}}
\pgfpathmoveto{\pgfpoint{\pgf@xa}{\pgf@yb}}
\pgfpathlineto{\pgfpoint{\pgf@xb}{\pgf@yb}}
 }
}
\makeatother

\newcommand{\riham}[1]{{\color{red}{#1}}}
\newcommand{\james}[1]{{\color{blue}{#1}}}


\begin{document}

\title{Bi-Directional A*}
\author{
\IEEEauthorblockN{Group 21}
\IEEEauthorblockA{Haoyang Zhang\\ Dhruvin Patel\\
Deep Pandya}
%\and
%\IEEEauthorblockN{Alan Turing}
%\IEEEauthorblockA{Rutgers University\\
% Piscataway, NJ, USA\\
% Email: alan1936@cs.rutgers.edu}
%\and
%%\IEEEauthorblockN{Third group member}
%\IEEEauthorblockA{Rutgers University\\
%\Piscataway, NJ, USA\\
%Email: Third_group_member@scarletmail.rutgers.edu}
}

\maketitle
\begin{abstract}
\textnormal{This project looks at the implementation of bidirectional A*. The project will visualize Bidirectional A* and the advantages of it over the unidirectional A*. The algorithm will be visualized in a clear manner which will be easy to understand. The Bi-Directional A* algorithm is relatively lesser known as compared to other search algorithms such as unidirectional A*, Dijkstra's, Iterative Deepening A* algorithms.}
\end{abstract}
%\onecolumn \maketitle \normalsize \vfill

\IEEEpeerreviewmaketitle
%%%%%%%%%%%%%%%%%%%%%%%%%%%%%%%%%%%%%%%%%%%%%%%%%%%%%%%%%%%%%%%%%%%%%%%%%%%%%%%%%%%%%%%%%%%%%%%%%%%%%%%%%
\section{Project Description}\label{sec:1. Project Description}
%%%%%%%%%%%%%%%%%%%%%%%%%%%%%%%%%%%%%%%%%%%%%%%%%%%%%%%%%%%%%%%%%%%%%%%%%%%%%%%%%%%%%%%%%%%%%%%%%%%%%%%%%
{This project aims at the visualization of the Bi-Directional A* Algorithm. This is to be achieved through an executable file in which the user will draw a graph arbitrarily using nodes and edges. The graph will contain a start node ``S'' and a goal node ``G''. For each node, the user will input the heuristic value for that particular node to ``S'' and ``G''. Similarly, for each edge the user will input the weight for the respective edge. Implementation of this algorithm lets you know the shortest path from start node to the goal node. This idea can be extended for finding shortest paths on maps, the closeness of the relationship between two persons on social networks, etc. The project is also useful to show why in some cases the Bi-Directional A* Algorithm will fail to return shortest path between ``S'' and ``G''. 
The successful completion of project is feasible in the given time frame as proposed along with a possibility of the addition of another bi-directional search in the form of Bi-Directional Dijkstra's algorithm as well as unidirectional A* and Dijkstra's algorithm. The Bi-Directional A* algorithm is relatively lesser known as compared to other search algorithms such as unidirectional A*, Dijkstra's, Iterative Deepening A* algorithms. The roadblocks for this project are:
\begin{itemize}{ 
\item
{Modification of heuristics for both paths from ``S'' to ``G'' and ``G'' to ``S'' so that we can meet our ``goal'' which is not the goal node ``G'' but the intersection of both paths.}
\item{A* is a unidirectional search, whereas the Bi-Directional A* will simultaneously perform two searches, i.e. from ``S'' to ``G'' and ``G'' to ``S''. in some cases , it is possible that these two searches miss each other completely and fail to meet somewhere in between ``S'' and ``G'' resulting in two paths leading to twice as consumption of time and space.}
\item{Checking if the path returned if optimal or not. If it is not optimal, then can there be an algorithm to modify the path in order to make it optimal.}
\item{Design of an appropriate UI to interact with the user.}

\end{itemsize}
Improving the Bi-Directional A* Algorithm to return an optimal path, dealing with the roadblocks for the algorithm and learning to program the GUI are some of the major challenges for this project. The time allocated to learn GUI programming would help gain experience for the team on GUI programming to make a better interface for the user. Utilizing the time for brainstorming another algorithm to implement may also be useful to come up with an optimized version of Bi-Directional A* that would return an optimal path.
}


%\subsection{Stage1 - The Requirement Gathering Stage. } \label{sec:1	Requirement Gathering Stage. } 
%\textnormal{
Get a realistic project idea that includes potential real world scenarios,
with a description of the different user types along with their interactions with the system 
as well as the system feedback to them, according to their information needs. 
This stage also requires the specification of the different constraints and restrictions
that need to be enforced depending on the different types of user (system interactions). 
The deliverables for this stage include the following items:
} 

\begin{itemize} 
\item{A general description (in plain English) of your project's deliverables (understandable by computer illiterate users).} 

\item{ A specific description of at least three types of users. }
	
\item{ A description of detailed real world scenarios (at least 2 scenarios) representing those typical interactions between the different user types and the system (including inputs and outputs and data types).}
	

\item{A description of detailed real world scenarios (at least 2 scenarios) representing those typical interactions between the different user types and the system (including inputs and outputs and data types). }

\item{A detailed time line for completion of the major implementaion stages together with the division of labor including testing, documentation, evaluaton, project report, and power point presentation.}

\end{itemize}

Please insert your deliverables for Stage1 as follows:

\begin{itemize} 
\item{The general system description: } 
Please insert the system description here.
\item{The three types of users (grouped by their data access/update rights): }
Please insert the users types in here, as follows:
\item{The user's interaction modes: }
Please insert the user's interaction modes here.
\item{The real world scenarios: }
Please insert the real world scenarios in here, as follows.
	\begin{itemize} 
	\item{Scenario1 description: }
	Please insert Scenario1 description here.
	\item{System Data Input for Scenario1: }
	Please insert System Data Input for Scenario1 here.
	\item{Input Data Types for Scenario1: }
	Please insert Input Data Types for Scenario1 here.
	\item{System Data Output for Scenario1: }
	Please insert System Data Output for Scenario1 here.
	\item{Output Data Types for Scenario1: }
	Please insert Output Data Types for Scenario1 in here.
	\item {Please repeat that pattern for each scenario (at least 2 scenarios per user).}
	\end{itemize}
Please repeat that pattern for each user type.
\item{Project Time line and Divison of Labor.}
Please insert here the time line and the corresponding implementation tasks.
\end{itemize}


\subsection{Stage1 - The Requirement Gathering Stage. }\label{sec:1 Requirement Gathering Stage. }
%%%%%%%
\textnormal{
Get a realistic project idea that includes potential real world scenarios,
with a description of the different user types along with their interactions with the system 
as well as the system feedback to them, according to their information needs. 
This stage also requires the specification of the different constraints and restrictions
that need to be enforced depending on the different types of user (system interactions). 
The deliverables for this stage include the following items:
} 

\begin{itemize} 
\item{A general description (in plain English) of your project's deliverables (understandable by computer illiterate users).} 

\item{ A specific description of at least three types of users. }
	
\item{ A description of detailed real world scenarios (at least 2 scenarios) representing those typical interactions between the different user types and the system (including inputs and outputs and data types).}
	

\item{A description of detailed real world scenarios (at least 2 scenarios) representing those typical interactions between the different user types and the system (including inputs and outputs and data types). }

\item{A detailed time line for completion of the major implementaion stages together with the division of labor including testing, documentation, evaluaton, project report, and power point presentation.}

\end{itemize}

Please insert your deliverables for Stage1 as follows:

\begin{itemize} 
\item{The general system description: } 
Please insert the system description here.
\item{The three types of users (grouped by their data access/update rights): }
Please insert the users types in here, as follows:
\item{The user's interaction modes: }
Please insert the user's interaction modes here.
\item{The real world scenarios: }
Please insert the real world scenarios in here, as follows.
	\begin{itemize} 
	\item{Scenario1 description: }
	Please insert Scenario1 description here.
	\item{System Data Input for Scenario1: }
	Please insert System Data Input for Scenario1 here.
	\item{Input Data Types for Scenario1: }
	Please insert Input Data Types for Scenario1 here.
	\item{System Data Output for Scenario1: }
	Please insert System Data Output for Scenario1 here.
	\item{Output Data Types for Scenario1: }
	Please insert Output Data Types for Scenario1 in here.
	\item {Please repeat that pattern for each scenario (at least 2 scenarios per user).}
	\end{itemize}
Please repeat that pattern for each user type.
\item{Project Time line and Divison of Labor.}
Please insert here the time line and the corresponding implementation tasks.
\end{itemize}


\subsection{Stage2 - The Design stage. }
\textnormal{
Transform the project requirements into a system flow diagram, specifyng the different algorithms, data types and structures required for processing and their associated operations.  
The deliverables for this stage include the system flow diagram containing a graphical representation and  textual descriptions of the corresponding data trasnformations, high level pseudo code of the overall system operation, and overall system time and space complexity.
}

%\begin{itemize} 
%\item{ }
%A brief textual description of the overall flow diagram (along with its functional operation in the different user scenarios described in the first stage of the project).
%\item{ }
%A specification of each algorithm and associated data structures together with its entities, attributes, and operations ( include an English description of how they relate to your user scenario(s)).

%\end{itemize}
Please insert your deliverables for Stage2 as follows:
\begin{itemize} 
\item{  Short Textual Project Description. }
Please insert here the flow diagram textual description here together with its overall time and space complexity.
\item{ Flow Diagram. }
Please insert your system Flow Diagram here.
\item{ High Level Pseudo Code System Description. }
Please insert high level pseudo-code describing the major system modules as per your flow diagram.
\item{Algorithms and  Data Structures. }
Please insert a brief description of each major Algorithm and its associated data structures here.
\end{itemize}

\begin{itemize} 
\item{  Flow Diagram Major Constraints.}
Please insert here the integrity constraints:
	\begin{itemize} 
	\item{ Integrity Constraint. }
	Please insert the first integrity constraint in here together with its description and justification. 
	\end{itemize}
Please repeat the pattern for each integrity constraint.
\end{itemize}


\subsection{Stage3- The Implementation Stage}
\textnormal{
Specify the language and programming environemnt you used for your implementation.
%Building the corresponding relational tables, according to the proposed ER model described in the previous phase %enforcing the different integrity constraints.  
The deliverables for this stage include the following items:
\begin{itemize} 
\item{}
Sample small data snippet. 
%The SQL tables that represent the ER project model, along with at least 3-5 rows of concrete data per table.
\item{}
Sample small output
%The normalization steps for each table, along with explanations/justifications of each normalization step.
\item{}
Working code
%The SQL table after the normalization steps (showing all table attributes).
\item{}
Demo and sample findings
%The SQL statements used to create the SQL tables, including the required triggers as well as the integrity constraints. At %least 2 triggers and 2 of each of the following constraint types have to exist in the project tables overall: 
\begin{itemize} 
\item{}
	Data size: In terms of  RAM size;  Disk Resident?; Streaming ?;  
\item{}
	List the most interestng findings in the data if it is a Data Exploration Project. For other project types consult with your project supervisor what the corresponding outcomes shall be. Concentrate on demonstrating the Usefuness and Novelty of your application.
%Whether some users will be denied access and/or updates to some data according to their roles (for example: student1 %can not access other students' ' grades, so a violation error pops up upon that action. Another example: a sales person %can see an item price, but can not change it, since only a manger can, also a violation error pops up upon that update %attempt).
\end{itemize}
\end{itemize}
}

\end{document}


