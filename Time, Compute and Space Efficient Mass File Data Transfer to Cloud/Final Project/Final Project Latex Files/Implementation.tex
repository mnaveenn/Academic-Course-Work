\textnormal{
For implementing Hufmann source coding algorithm, we have used Python 3 Version 3. We implemented the algorithms on HP Pavilion Notebook with Intel(R) Core(TM) i5-6200 CPU @2.30Ghz processor, 64bit operating system and 8 GB RAM.  
\\
The data for the Huffman coding algorithm was taken from github repository and it has text document containing all characters of the Multilingual European Subsets of Unicode and some other common Unicode subsets. This file contains about 466k English words. We will refer this file as `data.txt' / master data file. This file is used to generate the Huffman encoding table for compression of all files of similar type. For our implementation we have created 10 sample files of size around 2000 bytes. All the sample files were named in sequence from `sample1'.txt to `sample10.txt'. All the above samples files were taken from [2]. This files were placed in a single folder along with the python codes and the data.txt file. We have our implementation on two python files: `Huffman\_User\_Command.ipynb' and `Huffman\_Encoding\_demo.ipynb'.
The flow charts and diagrams were created using draw.io application 
}




