
In this project, there will be a blank screen in the application on which the user will draw a graph using nodes and edges and input the edge weights as well as heuristic for each node. The application will then run the algorithm for the user showing step-by-step traversal from start node to goal node.The deliverables for this stage include the following items:
\begin{itemize} 

	
\item{Academics
The teacher will have a sample graph saved in a .pkl and will upload this file in the application. The application will run the Bi-Directional A* on that graph. The teacher may also want to use a sample graph drawn on the application screen for educational purposes in future. For doing so, the teacher can extract a .pkl file for the respective graph from the application on the click of a button. The student will input a graph and will run the application in order to know the mechanism of the algorithm in a clear way. 
\begin{itemize}
\item{ Scenario: A typical example of a real world scenario can be taken in the form of a classroom. A teacher will show a sample graph and render the algorithm on that graph to show the students the logic behind Bi-Directional A* algorithm. The application developed for this project will demonstrate how the algorithm works. A student may ask to change the graph a little and observe the impact of change on the algorithm. This too can be accommodated in the application file.}
\item{Input: The graph in a pkl format or can draw a graph. }
\item{Output: Visualization of each step of BDA* along with the path. 
}
}
\end{itemize}	
{
\item{Researcher: A person who wants to compare different graph search algorithms and wants to find the optimal algorithm their work can make use of this application.
\begin{itemize}
    \item {Scenario: The researcher inputs a graph for which they have performed other search algorithms and will compare them to the result of our application. The researcher will know each step and thus can make improvements for the research purposes. }
    \item{Input: The graph in pkl format or can draw a graph.}
    \item{Output: Visualization of each step of BDA* along with the path.}
    \end{itemize}

}

}
\end{itemize}
{Timeline for the project:

\begin{itemize} 
\item{October 21st to October 27th: Complete the description of the project} 

\item{October 27th to November 2nd: Learn GUI Programming}

\item{November 2nd to November 9th: Draft a pseudo code and prepare a flow diagram for the project}
\item{November 9th to November 11th: Implement Bi-Directional A* Algorithm}
\item{November 11th to November 24th: Implement the GUI}
\item{November 24th to November 26th: Brainstorm for another algorithm to implement}
\item{November 26th to November 29th: Implement the new algorithm}
\item{November 29th to December 7th: Prepare the final report and presentation ppt.}
\end{itemize}
{Division of labor:
\begin{itemize}
    \item {Haoyang: Implementation and optimization of BDA* }
    \item {Deep: GUI and Report}
    \item {Dhruvin: GUI and report}
\end{itemize}
}