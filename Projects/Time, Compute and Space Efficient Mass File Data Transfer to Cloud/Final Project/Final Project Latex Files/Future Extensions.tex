\textnormal{
The possible future extensions for this project would be - 
\begin{itemize}
\item{One of the failures for the above approach is that this algorithm will fail when we encounter new characters that are not present in the Huffman coding tree. This problem can be solved taking an adaptive approach to the Huffman coding where in a new Huffman coding tree will be generated as when a new character is encountered in the data files. This will reduce the complexity of identifying a data file with all the characters for generating the Huffman coding tree.}
\item{Another possbile extension for this approach would be to improve the compression ratio in the above approach. This can be done creating more than one Huffman coding trees. These trees should have a significant difference in their distributions. Then we can compress our new file with all the trees and use the tree that gives us less compression ratio. Eventhough this method may not give us optimal compression but it will give a considerable amount of compression for most of the input files without increasing the running time complexity}
\end{itemize}
}