%%%%%%%%%%%%%%%%%%%%%%% file template.tex %%%%%%%%%%%%%%%%%%%%%%%%%
%
% This is a general template file for the LaTeX package SVJour3
% for Springer journals.          Springer Heidelberg 2010/09/16
%
% Copy it to a new file with a new name and use it as the basis
% for your article. Delete % signs as needed.
%
% This template includes a few options for different layouts and
% content for various journals. Please consult a previous issue of
% your journal as needed.
%
%%%%%%%%%%%%%%%%%%%%%%%%%%%%%%%%%%%%%%%%%%%%%%%%%%%%%%%%%%%%%%%%%%%
%
% First comes an example EPS file -- just ignore it and
% proceed on the \documentclass line
% your LaTeX will extract the file if required
\begin{filecontents*}{example.eps}
%!PS-Adobe-3.0 EPSF-3.0
%%BoundingBox: 19 19 221 221
%%CreationDate: Mon Sep 29 1997
%%Creator: programmed by hand (JK)
%%EndComments
gsave
newpath
  20 20 moveto
  20 220 lineto
  220 220 lineto
  220 20 lineto
closepath
2 setlinewidth
gsave
  .4 setgray fill
grestore
stroke
grestore
\end{filecontents*}
%
\RequirePackage{fix-cm}
%
%\documentclass{svjour3}                     % onecolumn (standard format)
%\documentclass[smallcondensed]{svjour3}     % onecolumn (ditto)
\documentclass[smallextended]{svjour3}       % onecolumn (second format)
%\documentclass[twocolumn]{svjour3}          % twocolumn
%
\smartqed  % flush right qed marks, e.g. at end of proof
%
\usepackage{graphicx}
%
% \usepackage{mathptmx}      % use Times fonts if available on your TeX system
%
% insert here the call for the packages your document requires
%\usepackage{latexsym}
% etc.
%
% please place your own definitions here and don't use \def but
% \newcommand{}{}
%
% Insert the name of "your journal" with
\journalname{Circuits, Systems,and Signal Processing}
%
\begin{document}

\title{Hardware Implementation of ECG Denoising System Using TMS320C6713 DSP Processor}
%\subtitle{Do you have a subtitle?\\ If so, write it here}

\titlerunning{ECG Denoising System Using TMS320C6713 DSP Processor}        % if too long for running head

\author{S. A. Anapagamini,         R. Rajavel %etc.
}

%\authorrunning{Short form of author list} % if too long for running head

\institute{S. A. Anapagamini \at
              Anand Institute of Higher Technology, \\
              Kalasalingam Nagar, IT Corridor,\\
              Old Mahablipuram Road, Chennai-603103.\\
              \email{gaminiajith90@gmail.com}           %  \\
%             \emph{Present address:} of F. Author  %  if needed
           \and
           R. Rajavel \at
           SSN College of Engineering, \\
					 Old Mahabalipuram Road, Chennai - 603 110
}

\date{Received: date / Accepted: date}
% The correct dates will be entered by the editor


\maketitle

\begin{abstract}
The Electrocardiogram (ECG) signal is extensively used biomedical signal for diagnosis of heart diseases. However the quality of ECG signal is deteriorated by several noises during its acquisition. The two dominant and recurring noises are power line interference and baseline wander noise and they have to be removed for better clinical evaluation. In this paper, an ECG denoising system is realized in TMS320C6713 DSP processor using a combination of Empirical Mode Decomposition (EMD) algorithm and FFT based frequency analysis. First the proposed denoising system is simulated and validated using MATLAB. Then, it is implemented in TMS320C6713 using Code Composer Studio (CCS). The emulation results indicate the successful removal of power line interference and baseline wander noise from the noisy ECG signal.
\keywords{ECG denoising system \and Powerline interference \and Empirical Mode Decomposition \and Baseline wander noise  \and FFT analysis}
% \PACS{PACS code1 \and PACS code2 \and more}
% \subclass{MSC code1 \and MSC code2 \and more}
\end{abstract}

\section{Introduction}
\label{Sec1}
The ECG signal is the representation of the electrical activity of the heart and aids the physiologist to diagnose the cardiac diseases. The ECG signal recorded during acquisition is often corrupted by different types of noise such as  power line interference, electrode contact noise, baseline wander noise, noise generated by electronic devices used in signal processing, and noise produced from the external electrical interference. The recurring noises that often cause misinterpretation and difficulty in diagnosis are power line interference and baseline wander noise. Power line interference is a high frequency noise (50 Hz noise or 60 Hz) and it occurs due to the electromagnetic interference from the electric-power system. Baseline wander, a very low frequency noise occurs due to respiration, and physical movements of the patients. These noises have to be filtered from the valid ECG for easy visual interpretation. Many offline based approaches have been reported in literature to remove noises in ECG such as IIR filter \cite{Mahes-Jour,Mapreet-Conf}, FIR filter \cite{Mapreet-Conf}, adaptive filter \cite{Hamilton-Jour}, wavelet transform \cite{Ashfanoor-Conf,Mapreet-Conf} and EMD algorithm \cite{Alexandros-Jour,Amit-Conf,Ashfanoor-Conf,Manuel-Jour,Na-Conf,Anapagamini-Conf}. The EMD is adaptive method with basic functions derived fully from the data and it does not require any priori known basis like Fourier, and wavelet transforms and reference signal like adaptive filter. So researchers used EMD technique widely to process non linear biomedical signals like ECG. 

The ECG denoising system uses EMD and low pass FIR filter to remove power line interference. The baseline wander noise is eliminated using a combination of both EMD and FFT based frequency analysis. The denoising system is first implemented in MATLAB and evaluated by means of both visual assessment and qualitatively in terms of Root Mean Square and Correlation Coefficient. The denoising system realized in MATLAB is implemented in TMS320C6713 DSP processor. EMD is batch processing algorithm which consumes many iteration and has heavy computational load. TMS320C6713 DSP processor is chosen for implementation because it is based on very long instruction word architecture and suitable for numerical intensive algorithm like EMD \cite{RulphChassing-Book}. The original ECG signal for this implementation is obtained from Massachusetts Institute of Technology-Beth Israel Hospital (MIT-BIH) database. Then the noisy ECG signal is obtained by manually adding the artifacts namely baseline wander noise (0.18 Hz) and power line interference (50Hz). 

The paper is organized as follows. Section 2 presents the concept of EMD algorithm. Section 3 briefly describes the realization of ECG denoising system in MATLAB. The performance and computational complexity of ECG denoising system is discussed in section 4. Section 5 describes the implementation of ECG denoising system in TMS320C6713 DSP processor. Frame based ECG denoising system and the experimental results are discussed in section 6. Section 7 presents the conclusion and future work


\section{EMD Algorithm}
\label{Sec2}
The Empirical Mode Decomposition (EMD) was proposed by Huang et al. \cite{George-Jour} as a tool to adaptively decompose a signal into a collection of AM-FM components. EMD behaves both like wavelet and dyadic filter bank for fractional gaussian noise  \cite{Huang-Conf}. Traditional data analysis methods, like Fourier and wavelet-based methods require some predefined basis functions to represent a signal. But EMD relies fully on data-driven mechanism that does not require any priori known basis. It is especially well suited for nonlinear and non-stationary signals, such as biomedical signals. The key feature of EMD is to decompose a signal into sum of intrinsic mode functions (IMF) with a final residue. The IMFs are estimated via an iterative procedure called sifting process. The IMF should satisfy the following conditions.\vspace*{0.1cm}
\\1. In the whole data set, the number of extrema and the number of zero crossings must be either equal or differ at most by one.\vspace*{0.1cm}
\\2. At any point, the mean value of the envelope defined by the local maxima and local minima should be zero \cite{Huang-Conf}.\vspace*{0.1cm}

The IMF includes different frequency band from high to low, with final residue. The sum of IMF and residue produced by sifting process should match the signal ensuring correctness of the algorithm. The EMD decomposes the ECG signal in to high frequency component (first IMF) and low frequency component (residue). The low frequency component is again decomposed in to high frequency component (second IMF) and a residue. This process continues until the residue becomes monotonic.  \\  
\\
The EMD algorithm can be summarised as follows \cite{Ming-Jour}\vspace*{0.2cm}
\\Step-1 $\:$ Identify all local maxima and local minima of the input signal.	\vspace*{0.15cm}
\\Step-2 $\:$ Generate upper envelope from local maxima and lower envelope from local minima using cubic spline interpolation.\vspace*{0.15cm}
\\Step-3 $\:$ Calculate the mean of upper and lower envelope \vspace*{0.15cm}
\\Step-4 $\:$ The mean is subtracted from the input signal to obtain IMF. The IMF should satisfy two conditions \vspace*{0.15cm}
\\ \hspace*{0.5cm} 1)	The number of extrema and the zero crossings should be equal or differ at most  by one \vspace*{0.15cm}
\\ \hspace*{0.5cm} 2)	The average value of the envelope defined by the local maxima and local minima should be equal to zero at any point.\vspace*{0.15cm}
\\Step-5 $\:$ The process is iterated from Step-1 until IMF is produced by considering the output of Step-4 as input signal.\vspace*{0.15cm}
\\Step-6 $\:$ Once the IMF is produced, it is subtracted from the input signal to get residue signal.\vspace*{0.15cm}
\\Step-7 $\:$ The residue is considered as the input signal and the process form Step-1 is continued until residue satisfies the the condition that, the number of extrema in residue should be less than two or the residue should be monotonic.


\section{ECG Denoising System}
\label{Sec3}
The ECG denoising system shown in Fig. 1 consist of EMD block which decompose the ECG signal into IMFs and residue, power line interference removal (PIR) block which removes power line interference and base detector block which estimates baseline wander noise and removes it. The original ECG signal x(t) obtained from MIT-BIH database  is first added with noises namely baseline wander noise and power line interference to generate noisy ECG signal. Then the noisy ECG signal is sent to EMD block which decomposes it in to various IMFS and residue.
\begin{figure}%[here]
\begin{center}
%\includegraphics[height=5.1cm, width=15cm]{ECG_Denoising_System.eps}
\includegraphics{Figure1.eps}
\label{Figure1} 
\caption{Proposed ECG denoising system}
\end{center}
\end{figure}

\subsection{Removal of Powerline Interference}
\label{Sec3.1}
EMD decomposition is a process which continuously filters high frequency components into low frequency component so power line interference which is an high frequency component appears in first IMF itself. Thus power line interference is removed by subtracting first IMF from noisy ECG signal. The removal of first IMF can cause issue such as attenuation in the QRS complex of the ECG signal. This attenuation is mainly due to the removal of some useful component of ECG signal along with the power line interference in first IMF. So first IMF is passed over low pass FIR filter with cut off frequency of 31 Hz to regain the useful component and it is added to compensate the effect of distortion. In this way the power line interference is removed from the noisy ECG signal without attenuating much of the original ECG signal. 

\subsection{Removal of Baseline Wander Noise }
\label{Sec3.2}	
The partially denoised ECG signal p(t) which still consist of baseline wander noise is passed through base detector block to estimate and remove baseline wander noise. The baseline wander noise which occurs in the frequency range of 0.15Hz to 0.3Hz get distributed in residue and higher order IMFs. In order to estimate the number of IMFs affected by baseline wander noise a strategy is proposed and is explained in flowchart shown in Fig.2. In the flowchart $L$ denotes the number of IMFS produced by sifting process. 

The base detector block first measures the FFT of each IMF in order to find its frequency. The IMFs which has frequency less than 0.5 Hz is added with residue signal to estimate baseline wander noise. The estimated baseline wander noise is removed from p(t) to obtain denoised ECG signal.  

\begin{figure}%[here]
\begin{center}
%\includegraphics[height=5.1cm, width=15cm]{ECG_Denoising_System.eps}
\includegraphics{Figure2.eps}
\label{Figure2} 
\caption{Flowchart representation of base detector}
\end{center}
\end{figure}

\subsection{Simulation Results}
\label{Sec3.3}
The original ECG signal collected from MIT-BIH database and noisy ECG signal which consists of baseline wander noise and power line interference is shown in Fig. 3 and Fig. 4. The noisy ECG signal is passed through denoising system implemented in MATLAB where power line interference and baseline wander noise are removed. Fig. 5 shows the compasion of original ECG signal and Denoised ECG signal. 

\begin{figure}%[here]
\begin{center}
\includegraphics[height=5.1cm, width=12cm]{Figure3.eps}
\label{Figure3} 
\caption{Original ECG signal-106 [Courtesy: MIT-BIH Database]}
\end{center}
\end{figure}

\begin{figure}%[here]
\begin{center}
\includegraphics[height=5.1cm, width=12cm]{Figure4.eps}
\label{Figure4} 
\caption{Noisy ECG signal}
\end{center}
\end{figure}

\begin{figure}%[here]
\begin{center}
\includegraphics[height=5.1cm, width=12cm]{Figure5.eps}
\label{Figure5} 
\caption{Comparison of reference ECG signal and denoised ECG signal}
\end{center}
\end{figure}



The blue dotted line shows the reference ECG signal and red solid line shows the denoised ECG signal. From the Fig. 5 it is clear that power line interference and baseline wander noise has been greatly reduced and denoised ECG signal has closer resemblance to original ECG signal.


\section{Performance Analysis of ECG Denoising System}
\label{Sec4}
A Combination of band pass and  high pass FIR filter of order 400 is designed \cite{Mapreet-Conf} and the performance is compared qualitatively in terms of Root Mean Square error (RMSE) and Correlation coefficient.

\begin{equation}
\label{RMSE eqn}
RMSE = \sqrt{\frac{\sum_{t=0}^{L-1}(x(t)-\hat{x}(t))^2}{L}}
\end{equation}

\begin{equation}
\label{Corr eqn}
r = \frac{n(\sum xy)-(\sum x)(\sum y)}{\sqrt{[n\sum x^2-(\sum x)^2][n\sum y^2 - (\sum y)^2]}}
\end{equation} \vspace*{0.15cm}

Where $x(t)$ is the reference ECG signal, $\hat{x}(t)$ is the denoised ECG signal and $L$ is the number of samples in equation (1) and $x$ represents reference ECG signal, $y$ represents denoised ECG signal and $n$ represents the number of samples in equation (2).
The RMSE and Correlation Coefficient is calculated for both filter based method and EMD based method and is shown in Table I. From Table I it is evident that EMD performs better than traditional filtering approach. 

% For tables use
\begin{table}
% table caption is above the table
\caption{Performance analysis of FIR filter approach and proposed EMD based approach}
\label{Table:1}       % Give a unique label
% For LaTeX tables use
\begin{tabular}{||c|c c|c c||}
\hline\noalign{\smallskip} \vspace*{0.15cm}
MIT-BIH DATA   & \ \ \ \ \ \ \ \ \ RMSE (mV)       &                      & \ \ \ \ CORRELATION   &   \\
REFERENCE NO:  &                 &                      & \ \ \ \ COFEEICIENT   &   \\
\hline\noalign{\smallskip} \vspace*{0.15cm}
               & PROPOSED        & FILTER               & PROPOSED      & FILTER  \\
               & METHOD          & METHOD               & METHOD        & METHOD  \\
\hline\noalign{\smallskip}  \vspace*{0.12cm}             
106            & 11.9            & 23.2                 & 0.9884        & 0.956  \\ \vspace*{0.15cm}
111            & 6.87            & 16.3                 & 0.9883        & 0.9290  \\
121            & 5.58            & 15.5                 & 0.9902        & 0.9356  \\
\noalign{\smallskip}\hline
\end{tabular}
\end{table}

The computational complexity is measured in terms of number of multiplication since multiplication operation requires more computation than addition. The filtering is basically a convolution operation. If the filter coefficient of length $N$ is convolved with data of length $L$ the number of multiplication involved is $N*L$. The computational complexity of EMD algorithm is $18*L + 15*M$ \cite{Julien-Jour} where $L$ represents the number of samples and $M$ represents the number of extrema (maxima and minima) obtained in $K$ iterations.  $K$ is the total number of iterations used to obtain all $I$ IMFS and residue in EMD algorithm. In addition to EMD algorithm a low pass filter of order 50 is used in power line interference removal and FFT based frequency analysis is used in baseline wander estimation. The number of complex multiplication involved in FFT algorithm is $(L/2)*log2L$. Hence the total computational complexity of ECG denoising system is $18*L+15*M+51*L+(I*(L/2)*log2L)$.


\begin{table}
% table caption is above the table
\caption{Computational Complexity Analysis}
\label{Table:2}       % Give a unique label
% For LaTeX tables use
\begin{tabular}{||c|c c||}
\hline\noalign{\smallskip} \vspace*{0.15cm}
MIT-BIH DATA   & \ \ \ \ \ \ \ NO. OF MULTIPLICATIONS       &                       \\
REFERENCE NO:  &                              &                       \\
\hline\noalign{\smallskip} \vspace*{0.15cm}
               & PROPOSED                     & FILTER BASED           \\
               & METHOD                       & METHOD                \\
\hline\noalign{\smallskip}  \vspace*{0.12cm}             
106            & 1399815                      & 2406000             \\ 
111            & 1182705                      & 2406000                 \\
121            & 978465                       & 2406000                 \\
\noalign{\smallskip}\hline
\end{tabular}
\end{table}

\section{ECG Denoising System in TMS320C6713 DSP Processor}
\label{Sec5}
The entire ECG denoising system is realized in TMS320C6713 DSP Processor using Code Composer Studio (CCS). CCS provides an integrated development environment (IDE) for real - time digital signal processing applications based on the C programming language. It is used to develop and verify the implementation of ECG denoising system in DSP processor. First the denoising system is experimented with the ECG signal of a frame of length of 128 samples which spans approximately 0.4 seconds. The original ECG signal taken from MIT-BIH database (106) and noisy ECG signal constructed by manually adding baseline wander and power line interference is shown in Fig. 6 and Fig. 7 respectively.


\begin{figure}%[here]
\begin{center}
\includegraphics[height=5.1cm, width=12cm]{Figure6.eps}
\label{Figure6} 
\caption{A Frame of Original ECG signal}
\end{center}
\end{figure}

\begin{figure}%[here]
\begin{center}
\includegraphics[height=5.1cm, width=12cm]{Figure7.eps}
\label{Figure7} 
\caption{A Frame Noisy ECG signal}
\end{center}
\end{figure}

The noisy ECG signal is sent to ECG denoising system to remove power line interference and baseline wander noise as disscussed in section \ref{Sec3}.

\begin{figure}%[here]
\begin{center}
\includegraphics[height=5.1cm, width=12cm]{Figure8.eps}
\label{Figure8} 
\caption{A Frame of Denoised ECG signal}
\end{center}
\end{figure}

\begin{figure}%[here]
\begin{center}
\includegraphics[height=5.1cm, width=12cm]{Figure9.eps}
\label{Figure9} 
\caption{Comparison of denoisedd ECG signal and original ECG signal}
\end{center}
\end{figure}


Fig. 8 shows a frame of denoised ECG signal and Fig. 9 shows the comparison of denoised ECG signal and original ECG signal for a frame of 128 samples. The upper waveform in Fig. 9 shows the original ECG signal and lower waveform in Fig. 9 shows the denoised ECG signal. The lower waveform matches very well with the upper waveform justifying both power line  interference and  baseline wander noise are removed.

\section{Frame Based ECG Denoising System}
\label{Sec6}

In order to process large no of frames and to process the signal continuously frame based processing is introduced. Frame based processing divides continuous sequences of input and output samples into frames of N samples. Rather than processing one input sample at each sampling instant, a new frame of N input samples must be processed every N sampling instant. Frame - based processing consists of three distinct activities
\begin{itemize}
	\item Collect a new frame of input samples.
	\item Process previously collected samples.
	\item Output the previously processed samples~\cite{RulphChassing-Book}
\end{itemize}
	
The noisy ECG signal taken for processing is of length 3000, which consists of 23 cycles and spans approximately 10 seconds. It is divided in to frames of 128$(N)$ samples. The input samples are saved as coefficient file in on-chip memory. The new frame of ECG signal is collected in input buffer for first $N$ sampling instant. For second $N$ sampling instant the frame of sample collected previously are sent to ECG denoising system and a new frame of ECG signal is collected in input buffer. During next sampling instant the output collected from ECG denoising system are displayed in digital storage oscilloscope (DSO) and previously collected samples are sent to ECG denoising system. This process continues until all samples are processed. 


\subsection{Experimental Results}
\label{Sec6.1}
The ECG denoising system implemented in TMS320C6713 using CCS studio and Digital Storage Oscilloscope used for displaying results, form the complete experimental setup for ECG signal denoising is shown in Fig. 10. The frames of original ECG signal, noisy ECG signal and the denoised ECG signals are shown in Fig. 11, Fig. 12 and Fig. 13 respectively.  

\begin{figure}%[here]
\begin{center}
\includegraphics[height=5.1cm, width=12cm]{Figure10.eps}
\label{Figure10} 
\caption{Experimental Setup}
\end{center}
\end{figure}

\begin{figure}%[here]
\begin{center}
\includegraphics[height=5.1cm, width=12cm]{Figure11.eps}
\label{Figure11} 
\caption{Frames of ECG signal}
\end{center}
\end{figure}

\begin{figure}%[here]
\begin{center}
\includegraphics[height=5.1cm, width=12cm]{Figure12.eps}
\label{Figure12} 
\caption{Frames of Noisy ECG signal}
\end{center}
\end{figure}

\begin{figure}%[here]
\begin{center}
\includegraphics[height=5.1cm, width=12cm]{Figure13.eps}
\label{Figure13} 
\caption{Frames of Denoised ECG signal}
\end{center}
\end{figure}

From Fig. 11 and Fig. 13 it is clear that denoised ECG signal matches very well with the original ECG signal justifying both power line noise and baseline wander noises are removed. The correlation coefficient between denoised ECG signal and original ECG signal for 128 samples is measured qualitatively to examine ECG denoising system. The correlation coefficient measured is 0.98 proving that proposed ECG denoising system works efficiently. Fig. 14 shows the comparison of noisy ECG signal and denoised ECG signal. The white coloured waveform is the noisy ECG signal containing power line interference and baseline wander noise and the light blue coloured waveform is the denoised ECG signal.

\begin{figure}%[here]
\begin{center}
\includegraphics[height=5.1cm, width=12cm]{Figure14.eps}
\label{Figure14} 
\caption{Comparison of noisy ECG signal and denoised ECG signal}
\end{center}
\end{figure}

\begin{figure}%[here]
\begin{center}
\includegraphics[height=5.1cm, width=12cm]{Figure15.eps}
\label{Figure15} 
\caption{Comparison of original ECG signal and denoised ECG signal}
\end{center}
\end{figure}

In Fig. 15 the white coloured waveform indicates the denoised ECG signal and light blue coloured waveform shows the original ECG signal. From Fig. 15, it is clear that denoised ECG signal is similar to original ECG signal, which justifies that proposed ECG denoising system performs well.

\section{Conclusion and Future Work}
\label{Sec7}

A denoising system capable of removing power line noise and baseline wander noise from noisy ECG signal has been proposed. The proposed ECG denoising system is implemented in MATLAB and outperforms FIR filter based denoising system in terms of RMSE and Correlation Coefficient. The computational complexity was also analysed in terms of number of multiplication involved for both FIR based ECG denoising system and proposed ECG denoising system. The proposed ECG denoising system has less computational complexity when compared with FIR based ECG denoising system. Thus the proposed method proved its effectiveness in artifact elimination.The proposed ECG denoising was then implemented in TMS320C6713 DSP processor for samples of length 128 which spans approximately 0.4 seconds. Next, the same approach was extended for ECG signal of length 3000 samples, which spans approximately 10 seconds using frame based approach. The experimental results in terms of RMSE and correlation coefficient show the successful removal of power line interference and baseline wander noise from the noisy ECG signal. 
	As with any other research work, there is still a lot of room for improvement. Here few ideas are presented that could be the subject of future work.  In this work, on chip memory is used to hold the ECG signal. Since there is a limitation in the on chip memory it is not possible to store more than 3000 samples. This system has to be extended to process the ECG signal continuously either from computer memory or ECG data acquisition system. ECG signal acquisition system was developed for this work, but unfortunately the acquired ECG signal is not consistent. Hence there is a need to develop a feasible ECG data acquisition system to demonstrate the effectiveness of the proposed system.

%\begin{acknowledgements}
%If you'd like to thank anyone, place your comments here
%and remove the percent signs.
%\end{acknowledgements}

% BibTeX users please use one of
%\bibliographystyle{spbasic}      % basic style, author-year citations
%\bibliographystyle{spmpsci}      % mathematics and physical sciences
%\bibliographystyle{spphys}       % APS-like style for physics
%\bibliography{}   % name your BibTeX data base

% Non-BibTeX users please use
\begin{thebibliography}{}
%
% and use \bibitem to create references. Consult the Instructions
% for authors for reference list style.
%
\bibitem{Abde-Jour}
Abde Quahab Boundraa, and Jean-Christopher Cexus, EMD-Based Signal Filtering, IEEE Transactions On Instrumentations and Measurement, Vol. 56., No. 6., pp.2196-2202 (2007).
\\
\bibitem{Alexandros-Jour}
Alexandros Karagiannis, Philip Constantinou, Noise Assited Data Processing With Empirical Mode Decomposition in Biomedical Signals, IEEE Transactions on Information Technology in Biomedicine, Vol.15., No.1., pp.11-18 (2011).
\\
\bibitem{Amit-Conf}
Amit J Nimunkar, Willis J Tompkins, EMD-based 60-Hz noise filtering of the ECG, IEEE EMBS. Annual International Conference, Madison, pp.1904-1907(2007).
\\
\bibitem{Ashfanoor-Conf}
Ashfanoor Kabir and Celia Sahnaz, In ECG Signal Denoising Method Based On Enchancement Algorithm in EMD and Wavelet Domain, IEEE TENCON 2011, Dhaka, Bangladesh, pp.284-287(2011).
\\
\bibitem{Reddy-Book}
D C Reddy, Biomedical signal processing Principles and technique, Tata McGraw-Hill Publishing company limited (2005).
\\
\bibitem{Gabriel-Work}
Gabriel Rilling, Patrick and Paulo Gonclaes, On Empirical Mode Decomposition And Its Algorithms, IEEE/EURASIP workshop on Nonlinear Signal and Image Processing, Grado, Italy (2003). 
\\
\bibitem{George-Jour}
George B. Moody and Roger G. Mark, The Impact of the MIT-BIH Arrhythmia Database, IEEE Engineering in Medicine and Biology, Vol.1., pp.0739-517520 (2011).
\\
\bibitem{Huang-Conf}
N. E. Huang, Z. E.  Shen, M. C. Wu, H. H. Shih, Q. Zheng, N. C. Yen, C. C. Tung, and H. H. Liu, The empirical mode decomposition and the Hilbert spectrum for nonlinear and non-stationary time series analysis, Proceedings of royal Society. London. A Math. Phys. Eng. Sci., Vol.454., No.1971, pp.903-995 (1998).
\\
\bibitem{Hamilton-Jour}
P. S. Hamilton, A comparison of adaptive and non adaptive filters for reduction of power line interference in the ECG, IEEE Transactions on Biomedical Engineering, Vol.43., No.1., pp.105-109 (1996).
\\
\bibitem{Julien-Jour}
Julien Fleureau, Amar Kachenoura, Laurent Albera, Jean-Claude, Nunes, Lotfi  Senhadji1, Multivariate Empirical Mode Decomposition and Application To Multichannel Filtering, Signal Processing, Vol. 91, No. 12, pp. 2783�2792 (2011).
\\
\bibitem{Mahes-Jour}
Mahes S. Chavan, R. A. Agarwala, M. D. Vplane, Suppression of baseline wander and power line interference in ECG using digital IIR filter, International Journal Of Circuits, Systems And Signal Processing, Vol.2., pp.356-365 (2008).
\\
\bibitem{Manuel-Jour}
Manuel Blanco Velasco, Binwei Weng, and Kenneth E. Barner, ECG Signal Denoising and Baseline wander Correction Based on The Empirical Mode Decomposition, Computers in Biology and Medicine, Vol.38., pp. 1-13 (2008).
\\
\bibitem{Mapreet-Conf}
Mapreet Kaur, Birmohan Singh, Seema, Comparison of Different Approaches for Removal of Baseline Wander from ECG signal, International Conference and Workshop on Emerging Trends In Technology, Mumbai, India, pp.30-36 (2011).
\\
\bibitem{Ming-Jour}
Ming Huan Lee, Kuo- kai Shyu, Po-Lei Lee, and Yun-Jen Chiu, Hardware   Implementation of EMD Using DSP and FPGA for Online Signal Processing, IEEE Transaction  on Industrial Electronics , Vol.58., No.58., pp.0278-0046 (2011).
\\
\bibitem{Na-Conf}
Na pan Vai Mang, Mai Peng Un,Pun Sio Hang, Accurate Removal Of  Baseline Wander In ECG Using Empirical Mode Decomposition, Proceedings of NFSI \& ICFBI.,China, pp.177-180 2007).
\\
\bibitem{Patrick-Lett}
Patrick Flandrin, and Gabriel Rilling, Empirical mode decomposition as a filter bank, IEEE Signal Processing Letters, Vol. 11., No.2. (2004).
\\
\bibitem{RulphChassing-Book}
Rulph Chassing and Donald Reddy, Digital signal processing and application with the TMS320C6713 and TMS320C416 DSK, A John Wiley \& Sons, Inc., Publication, Second edition (2008).
\\
\bibitem{Zhao-Conf}
Z.Zhao,Yu-Quan, A New Method For Removal of Baseline Wander and Power line Interference in ECG signals, IEEE International Conference on Machine Learning, Hangzhou Dianzi Univ, Yu-Quan Chen,  Dalian, pp.4342-4347 (2006).
\\
\bibitem{Anapagamini-Conf}
S.A Anapagamini and R. Rajavel, Removal of Artifacts in ECG using Empirical Mode Decomposition, International Conference on Communication and Signal Processing, Adhiparasakthi Engineering College, Melmaruvathur, pp.845-849 (2013). 

\end{thebibliography}

\end{document}
% end of file template.tex

